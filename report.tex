\documentclass{article}
\usepackage[utf8]{inputenc}
\usepackage {mathtools, graphicx, amsfonts, amssymb, comment}
\usepackage{enumerate}
\usepackage[backend=bibtex,sorting=none]{biblatex}
\usepackage{listings}
\usepackage{hyperref}

\addbibresource{sources.bib}

\title{Finding Roots of Functions}
\author{Matt Gaikema}
\date{\today}

\begin{document}

\maketitle

\section{Introduction}

\section{Newton's Method}

\section{Analytic Functions}

\subsection{Argument Principle}

According to \cite{delves1967numerical},
if $C$ is a closed curve in the complex plane which does not pass through a zero of $f(z)$, and $R$ is in the interior of $C$, 
then
\begin{equation}\label{eqn:arg-princ}
	s_n=\frac{1}{2\pi i}\oint_Cz^N\frac{f'(z)}{f(z)}dz=\sum_{j=1}^vz_j^N,
\end{equation}
where $z_j,j\in\{1,2,\dots,v\}$ are the zeros of $f$ which lie in $R$.
A multiple zero is counted according to its multiplicity.
When $N=0$, equation \ref{eqn:arg-princ} gives the number of zeros in $C$.
This is a result of the argument principle.\cite{wiki:argument-principle}

\subsection{Numerically finding roots}

\cite{delves1967numerical} gives a method for numerically locating the zeros of an analytic function (hence the title).

\printbibliography

\end{document}