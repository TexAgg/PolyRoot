\documentclass{article}
\usepackage[utf8]{inputenc}
\usepackage {mathtools, graphicx, amsfonts, amssymb, comment}
\usepackage{enumerate}
\usepackage[backend=bibtex,sorting=none]{biblatex}
\usepackage{listings}
\usepackage{hyperref}

\addbibresource{sources.bib}

\title{Finding Roots of Polynomials}
\author{Matt Gaikema}
\date{\today}

\begin{document}

\maketitle

\section{Introduction}

\section{General Formulas}

If a polynomial is of degree 4 or less, the roots can easily (relatively) be found with explicit formulas.
Galois showed us that there is not an explicit formula for 5th degree polynomials and above.

\subsection{Quadratic Polynomials}

For polynomials of the form $ax^2+bx+c$, the roots are of the form
\begin{equation}
	x = \frac{-b\pm\sqrt{b^2-4ac}}{2a},
\end{equation}
as many people know from seventh or eighth grade. 

\subsection{Cubic Polynomials}

For a polynomial of the form $ax^3+bx^2+cx+d$, the formula for the roots is slightly more complex than that of quadratic polynomials.\cite{wiki:cubic}
\begin{equation}
	x_k=-\frac{1}{3a}\bigg(b+u_kC+\frac{\Delta_0}{u_kC}\bigg),\,k\in{1,2,3},
\end{equation}
where 
\[u_1=1,\,u_2=\frac{-1+i\sqrt{3}}{2},\,u_3=\frac{-1-i\sqrt{3}}{2}\]

\section{Newton's Method}

\section{Analytic Functions}

\subsection{Argument Principle}

According to \cite{delves1967numerical},
if $C$ is a closed curve in the complex plane which does not pass through a zero of $f(z)$, and $R$ is in the interior of $C$, 
then
\begin{equation}\label{eqn:arg-princ}
	s_n=\frac{1}{2\pi i}\oint_Cz^N\frac{f'(z)}{f(z)}dz=\sum_{j=1}^vz_j^N,
\end{equation}
where $z_j,j\in\{1,2,\dots,v\}$ are the zeros of $f$ which lie in $R$.
A multiple zero is counted according to its multiplicity.
When $N=0$, equation \ref{eqn:arg-princ} gives the number of zeros in $C$.
This is a result of the argument principle.\cite{wiki:argument-principle}

Unfortunately, calculating $s_n$ using the Residue Theorem usually involves finding the roots anyways.

\subsection{Numerically finding roots}

\cite{delves1967numerical} gives a method, which I will refer to as the "Delves-Lyness Method", for numerically locating the zeros of an analytic function (hence the title),
which has four sections:
\begin{enumerate}
	% 1
	\item Evauluate the number of roots, $s_0=v$, in the region.
	If the number is manageble, calculate $s_1,s_2,\dots,s_v$, and carry on to step (3).
	Otherwise, continue.
	% 2
	\item
	Subdivide the region into smaller subregions and repeat step (1).
\end{enumerate}

\printbibliography

\end{document}